\section{Introdução}

\begin{frame}
\frametitle{Objetivo:}
Este trabalho visa apresentar um método para a classificação 
da linguagem corporal dos pacientes no leito.
Especificamente, 
é proposto um sistema baseado numa rede neuronal 
convolucional para classificar a linguagem corporal entre 4 tipos de mensagens corporais:
\begin{itemize}
    \item Dor
    \item Estado (-) 
    \item Neutro
    \item Estado (+)
\end{itemize}
\end{frame}


\begin{frame}
    \frametitle{Objetivo:}
    Este trabalho visa apresentar um método para a classificação 
    da linguagem corporal dos pacientes no leito.
    Especificamente, 
    é proposto um sistema baseado numa rede neuronal 
    convolucional para classificar a linguagem corporal entre 4 tipos de mensagens corporais:
    \begin{figure}[!ht]
    \centering
    \caption{Fonte: dreamstime.com}
        \begin{minipage}[t]{0.23\textwidth}
        \centering
          \includegraphics[width=0.95\textwidth]{images/output-pain.jpg}
          \subcaption{Dor \label{fig:output:1}}
        \end{minipage}
        \hfill
        \begin{minipage}[t]{0.23\textwidth}
        \centering
          \includegraphics[width=0.95\textwidth]{images/output-sad.jpg}
          \subcaption{Estado (-) \label{fig:output:2}}
        \end{minipage}
        \hfill
        \begin{minipage}[t]{0.23\textwidth}
        \centering
          \includegraphics[width=0.95\textwidth]{images/output-neutral.jpg}
          \subcaption{Neutro \label{fig:output:3}}
        \end{minipage}
        \hfill
        \begin{minipage}[t]{0.23\textwidth}
        \centering
          \includegraphics[width=0.95\textwidth]{images/output-happy.jpg}
          \subcaption{Estado (+) \label{fig:output:4}}
        \end{minipage}
        \label{fig:output}
      \end{figure}
\end{frame}
    





